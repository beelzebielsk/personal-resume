\documentclass{article}
%- Preamble: {{{ -------------------------------------------

% Disable page numbers.
\pagenumbering{gobble}
%- Packages: -----------------------------------------------
\usepackage[margin=.7in, asymmetric, centering]{geometry}
\usepackage{float}
\usepackage{enumitem}
\usepackage{fontspec} % Requires LuaLaTeX!
\usepackage{array} % For extra table stuffs, like before/after column commands.
\usepackage{tabu}
%- Fonts: --------------------------------------------------
\setmainfont{Heuristica}
%- Lengths: ------------------------------------------------
\setlength{\parindent}{0pt}
%- Commands: {{{ -------------------------------------------
% To do small caps: \textsc{}.
\newcommand{\name}[1]{{\huge #1} \vspace{10pt}}
\newcommand{\sectionTitle}[1]{{\Large #1} \vspace{4pt}}
\newcommand{\subSectionTitle}[1]{%
	{\large \textbf{#1}} \vspace{4pt}%
}
\newcommand{\placeStyle}[1]{\textbf{#1}}
\newcommand{\positionStyle}[1]{\textit{#1}}
\newenvironment{newplace}
  {
    \setlength{\tabcolsep}{0pt}
    \begin{tabu} to \textwidth [h!]{X[65,l]X[35,r]}
  }
  {\end{tabu}}
\newcommand{\placerow}[2]{\placeStyle{#1} & #2\tabularnewline}
\newcommand{\jobrow}[2]{\positionStyle{#1} & #2\tabularnewline}
\newcommand{\plainrow}[2]{#1 & #2\tabularnewline}
\newenvironment{bullets}
	{\begin{itemize}[noitemsep, topsep=0pt]}
	{\end{itemize}}
\newenvironment{resumesection}[1]
	{\sectionTitle{#1}}
	{\vspace{10pt}}

%- }}} -----------------------------------------------------
%- }}} -----------------------------------------------------
\begin{document}

% TODO:
% - Make a table out of the top.
% - Remove the page numbers from the page, as they're not necessary
%   (we're not doing more than one page).
% - Consider putting the Single Cycle CPU and continuations project on
%   your resume.

% - What specific things do I want to communicate to an employer about
%   me?
%   - I can do research/I can take a problem, break it down,
%     investigate it and come up with a solution. I had to do this for
%     the CPU, for creating continuations, a bunch for MSK.
%   - I can come up with a concept for a project, which I brainstormed
%     with my hoopla team.
%   - I can collaborate a team and deliver a product under a deadline,
%     which I had to do in the hackathon (strict deadline, prioritize
%     working result over ambitious one) and with my hoopla team (less
%     strict deadline, deliver ambitious and working results).
%   - I can take some initiative and risk: I will do extra work if I'm
%     invested in something. I am curious about what I do, I like to
%     ask questions, and I'll put in the extra time necessary to get
%     satisfactory answers. I did that with continuations because I
%     finished continued to do work on the project after the semester
%     ended. I've done that with little extra assignments that I give
%     myself.
%   - I can follow a specification, which I had to do with MSK
%     (loosely) and for designing the CPUs.
%   - I can extend existing work: the continutations interpreter and
%     MSK work with the original codebase. 

\begin{center}
	\name{Adam Ibrahim}

  {\setlength{\tabcolsep}{0pt}
    \begin{tabu} to \textwidth {XX[r]}
      ibrahimadam193@gmail.com  &  github.com/beelzebielsk \\
      (347)-458-8082 &             www.linkedin.com/in/adam-ibrahim 
    \end{tabu}
  }
\end{center}

\begin{resumesection}{Education}

%\begin{newplace}
	%\placerow{New York City College of Technology}
			     %{Fall 2011 - Spring 2014}
	%\placerow{}
			     %{Brooklyn, NY}
%\end{newplace}
%
%\begin{bullets}
	%\item Pursued AS in Computer Science
%\end{bullets}

\begin{newplace}
  \placerow{The City University of New York, City College}
					 {Expected: August 2018}
  \jobrow  {Bachelors of Science in Computer Science}{}
\end{newplace}

\begin{newplace}
	\placerow{Relevant Coursework}{}
\end{newplace}

Assembly Programming,
Data Structures,
Algorithms,
%Intro to Theoretical Computer Science,
Software Engineering,
Programming Paradigms,
Database Systems,
Computer Organization,
Artificial Intelligence
Operating Systems,
Intro to Machine Learning,
Computer Security

%\begin{bullets}
	%\item CSc 21000: Assembly
	%\item CSc 21200: Data Structures
	%\item CSc 22000: Algorithms
	%\item CSc 30400: Intro to Theoretical Computer Science
	%\item CSc 32200: Software Engineering
	%\item CSc 33500: Programming Paradigms
	%\item CSc 33600: Database Systems
	%\item CSc 34200: Computer Organization
	%\item CSc 44800: Artificial Intelligence
  %\item CSc 33200: Operating Systems
  %\item CSc 59929: Intro to Machine Learning
  %\item CSc 48000: Computer Security
%\end{bullets}

\end{resumesection}

\begin{resumesection}{Technical Skills}

\begin{bullets}
  \item Skills/Technologies: Python, BASH, Javascript, Node.js, Git,
        HTML, CSS, C++, C, Scheme, VHDL
  \item Operating Systems: Linux, Windows
\end{bullets}

\end{resumesection}

% Place these in reverse chronological order, so that people start at
% what you're doing now, then can dive into a bit of your history.
\begin{resumesection}{Relevant Experience}

%- Hoopla: {{{ ---------------------------------------------

\begin{newplace}
  \placerow{Hoopla}{September 2017 - Present}
\end{newplace}

\begin{bullets}
  \item Develop website where users can share DIY ideas with a network
        of people, either suggesting new ideas or tackling a suggested
        project 
  \item Design website backend using Node.js and Express, alongside
        PostgreSQL 
\end{bullets}

%- }}} -----------------------------------------------------
%- CTP: {{{ ------------------------------------------------

\begin{newplace}
	\placerow{CUNY Tech Prep, Javascript}
				   {June 2017 - Present}
	\jobrow  {Software Developer}
					 {New York, NY}
\end{newplace}

\begin{bullets}
  \item Selected to competitive full stack JavaScript training program
        that prepares students for careers in software engineering
  \item Learn in-demand technologies like \textit{React} and
        \textit{Node} + \textit{Express} and processes for design,
        implementation, testing, and deployment
\end{bullets}

%- }}} -----------------------------------------------------
%- MSK: {{{ ------------------------------------------------

\begin{newplace}
  \placerow{Medical Imaging research, Memorial Sloan Kettering Cancer
            Center}
				   {Summer 2017}
  \jobrow  {Student Researcher}
				   {New York, NY}
\end{newplace}

\begin{bullets}
  \item Organized existing codebase for semi-automatic segmentation
        into a library
  \item Implemented segmentation algorithms from new research papers
  \item Learned new image processing/segmentation material
        independently during employment, while handling other duties
\end{bullets}

%- }}} -----------------------------------------------------
%- Interpreter: {{{ ----------------------------------------

\begin{newplace}
  \placerow{Continuation-Passing Scheme Interpreter, Scheme}
           {Spring 2017}
\end{newplace}

\begin{bullets}
  \item Extended existing interpreter with new functionality:
        creation and use of continutations, exception handling
\end{bullets}

%- }}} -----------------------------------------------------
%- RBC Hackathon: {{{ --------------------------------------

% Blog entry to visit:
% http://www.zahncenternyc.com/24-hr-hackathon-sparks-innovative-fintech-solutions/
\begin{newplace}
  \placerow{RBC Fintech Hackathon, Python}
					 {New York, NY}
	\jobrow  {3rd Place}
				   %{October 20th-21st, 2016}
				   {October 2016}
\end{newplace}

\begin{bullets}
  \item Collaborated with a team of three other developers, and two
        economists to create an AI Chatbot in a single night using
        unfamiliar technologies
  \item Prioritized goals to deliver best possible product with tight
        deadlines
\end{bullets}

%- }}} -----------------------------------------------------
%- CPU: {{{ ------------------------------------------------

\begin{newplace}
  \placerow{Single/Multi Cycle CPU, VHDL}
           {Spring 2016}
\end{newplace}

\begin{bullets}
  \item Implemented CPU on an FPGA using a hardware-description
        language
  \item Created product based on specifications on a two-week
        deadline
  \item Wrote sample program for CPU in CPU's machine code,
        demonstrating CPUs ability to branch
\end{bullets}

%- }}} -----------------------------------------------------
%- DIMACS: {{{ ---------------------------------------------

\begin{newplace}
	%\placerow{Graph Pebbling research at Rutger's center for Discrete
  %          Mathematics and Theoretical Computer Science (DIMACS)}
	%			   {Summer 2013}
	%\jobrow  {Student Researcher}
	%			   {Rutgers University, Busch Campus}
  % TODO: Change the default style to make all of the behavior like
  % this: left stuff on the left, right stuff on the right. 
    \begin{tabu} to \linewidth {X}
      \placeStyle{Graph Pebbling research, Rutger's center for
                  Discrete Mathematics and Theoretical Computer
                  Science (DIMACS)} 
      \\
      \positionStyle{Student Researcher}
    \end{tabu}
    &
    \begin{tabu} to \linewidth {X[r]}
      Summer 2013
      \\
      Rutgers Univerity, Busch Campus
    \end{tabu}
\end{newplace}

\begin{bullets}
  \item Completed research in algorithms and theoretical computer
        science
  \item Found shortest paths algorithm for Graph Pebbling and
        explained results in academic paper
\end{bullets}

%- }}} -----------------------------------------------------
%- Emerging Scholars (out): {{{ ----------------------------

%\begin{newplace}
	%\placerow{Emerging Scholar at NYCCT}
				   %{Fall 2012 - Spring 2014}
%\end{newplace}
%
%\begin{bullets}
	%\item Compiled reviews of 2-3 existing research papers
	%\item Presented findings to college community in poster sessions
%\end{bullets}

%- }}} -----------------------------------------------------

\end{resumesection}

\end{document}
